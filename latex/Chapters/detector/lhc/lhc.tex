\section{The Large Hadron Collider (LHC)}

The Large Hadron Collider (LHC) is located at the European Organisation for Nuclear Research (CERN) near Geneva in Switzerland. On the 23rd November 2009 the collider began producing collisions at a centre of mass energy of 900\gev. Over the course of its operation the energy of the collisions has increased, for proton-proton collisions the centre of mass energy was increased first to 7\TeV (28 February 2010) and then 8\TeV (5 April 2012) with a peak luminosity of the order $10^{34}\cm^{-2}\sec^{-1}$, making the LHC world's highest energy and luminosity particle accelerator. 

%As well as protons the LHC collides protons with heavy lead ions with dedicated runs in 2010 and 2011.

In the future the LHC intends to continue to push the energy and luminosity boundaries to reach unprecedented levels of high energy interactions and data collection. The goal being to produce collisions with a centre of mass energy of 14 TeV or greater. In addition to this, the sub-detector components will be upgraded to improve in areas such as detector readout rate and the resolution of kinematic quantities. Together with the increase in beam energy and luminosity this will give scientists at the LHC the ability to challenge the Standard Model at a level of greater detail than ever seen before.

Stationed at locations around the LHC are several detectors. The two counter-directional beams of proton bunches are focused at these positions such that a large number of collisions occur at the position of the detectors. The six main detectors, ALICE\cite{Aamodt:1129812}, ATLAS\cite{Aad:1129811}, CMS\cite{Chatrchyan:1129810}, LHCb\cite{Alves:1129809}, TOTEM\cite{Anelli:1129807} and LHCf\cite{Adriani:1129808} function as complementary experiments. A full discussion with regards to each detector at the LHC is beyond the scope of this thesis, therefore this thesis focuses on the most significant detector related to the research discussed, the LHCb detector.

% B production
% "LHC will be by far the most copious source of B mesons, due to the high $b\bar{b}$ cross section and high luminosity"
% "The LHCb experiment plans to operate with an average luminosity of $2 \times 10^{32}
% $10^{12} b\bar{b}$ pairs produced a year
% $b\bar{b}$ pairs are predominantly produced in the same forward cone. 
% Trigger selects B events based on particles with large transverse momentum and displaced decay vertices
% decay vertex resolution && particle Id, vertex resolution studying rapidly oscillating $B_s$ mesons 
% pion - kaon separation - Bd -> pi pi heavily contaminated by Bd -> Kpi, Bs -> Kpi and Bs->KK. => systematic errors in the measured CP asymmetry in Bd -> pipi decays.
% pion - kaon separation - Bs -> DsK, backgrounds: Bs -> Dspi (no CP violation is expected in Bs ->Dspi)
% < luminosity => < Detector occupancy && < radiation damage && < pile-up