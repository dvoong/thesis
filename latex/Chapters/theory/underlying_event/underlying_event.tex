\subsection{The Underlying Event}
\label{section: underlying event}

The underlying event is any other activity in an event that accompanies a hard process, it contains contributions from the beam remnants - the left over proton fragments after the hard scatter - multiple parton interactions and initial and final state radiation. The hard scatter consists of the two outgoing jets, the initial state radiation leading to the hard process and the particles originating from the hard final state radiation.

The beam remnants are particles that evolves from the remainder constituents of the beam particle that do not take part in the hard process. These may be colour connected to the hard process due to colour confinement e.g. for a proton-proton interaction, a proton that initiates a hard process via a quark initiator will have remaining constituents that form a colour triplet. The colour connections are later resolved during the hadronisation process which ensures the final state of the interaction is composed of colour singlet hadrons. 

%Pythia: Highly developed multiple interaction model
%HERWIG: A MPI model is built into Herwig++ but a separate module (JIMMY) has to be interfaced to the Fortran version

%Lorentz contraction of highly boosted beam particles in result in disc like shapes in the laboratory rest frame. Interacting partons from the beam particles are extremely localised such that the parton shower and subsequent hadronisation are also. An overlap between the other partons in the beam particles giving rise to the potential of Multiple Parton Interactions (MPI) each with their own associated parton shower and hadronization. The additional interactions in multiple parton interactions are dominated by soft interactions though contributions exists from hard and semi-hard interactions. 

% and have been shown to have a significant effect on the particle multiplicity of the event.


% Beam remnant: Particles which do not take an "active" part in the initial-state radiation or hard-scattering process.
% 	Is colour connected to the primary 
% Multiple interactions: When more than one parton from each beam particle have "significant" interactions. in a fraction of events these additional scatterings maybe hard or semi-hard, but mostly they are soft in comparison to the hard process
% 	Important to model the impact parameter structure of hadron-hadron collisions
%	Impact parameter: Measure of how much overlap there is between hadrons
%		If it is small there is a high probability of multiple interactions
%		if it is small there is a high probability of a hard scatter
%		thus if it is small there is on average a higher amount of underlying event activity
%		if it is large there is a small probability of multiple interactions
% Primordial $k_\perp$: Transverse momentum of the shower initiating parton - takes into account the motion of quarks inside the original hadron.
% Underlying event: Arises from collisions between partons in the incoming hadrons that do not directly participate in the hard subprocess
% 	Everything except the hard radiation from the matrix element in the hard process for a single particle collision
% 	Minimum bias data - not identical but still related. Used to study correlations...

% MPI
%The most common hard subprocess at a high-energy hadron collider, such as the LHC, is elastic gluon-gluon scattering, $gg \rightarrow gg$ . The leading-order differential cross section for this subprocess diverges at zero momentum transfer, due to the exchange of a massless virtual gluon. This indicates that there are many interactions between beam particles though of a soft nature. 

%Multiple parton interactions is a term used to describe a proton-proton collision in which more than one parton from each of the protons are involved in an interaction. In general a proton-proton interaction is described by its primary interaction i.e. the parton interaction at the highest interaction scale, relative to this additional interactions are soft scale interactions due to an asymptotically decreasing cross-section with the interaction scale. However the probability of additional interactions that are hard or semi-hard is significant and has considerable effects on distributions such as the particle multiplicity.

% Why is it important
%A rather mundane reason for this is that if two pairs of partons collide in a single proton-proton collision, this can lead to some very peculiar-looking events. Two "standard" collisions can gang up and look very non-standard, possibly leading you to think you have found some physics beyond the Standard Model. (Probably supersymmetry.) So if you want to claim this, you need some understanding of the probability of multiparton interactions occurring in a single proton-proton collision.
%
%But on top of that, the issue of how a strongly interacting quantum field theory gives rise to the proton, an ultra-stable bound state with a lifetime longer than the age of the universe, present in the heart of every atom of matter, is an exciting open question. It's a problem being tackled from several experimental, theoretical and computational directions. I think understanding correlations between partons in high energy collisions will make an important contribution to this.