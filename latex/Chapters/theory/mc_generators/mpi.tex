\subsection{Multiple Parton Interactions (MPIs)}
%Multiple parton interactions is a term used to describe a proton-proton collision in which more than one parton from each of the protons are involved in an interaction. In general a proton-proton interaction is described by its primary interaction i.e. the parton interaction at the highest interaction scale, relative to this additional interactions are soft scale interactions due to an asymptotically decreasing cross-section with the interaction scale. Though the probability of hard or semi-hard additional interactions are smaller than for soft interactions at higher energies this increases due to the ability of partons to resolve partons in the proton with lower longitudinal momentum fraction $x$.

Multiple parton interactions is a term used to describe a proton-proton collision in which there is more than one hard scatter between the constituent partons. Observations of MPI effects can be seen in data from hadron collisions at the Intersecting Storage Rings (ISR) at CERN \cite{Akesson:1986iv} and the Fermilab Tevatron collider \cite{Abe:1997bp} \cite{Drees:1996rw} \cite{Abazov:2009gc}. Soft MPI effects have been observed in proton-proton collisions at Collider Detector Fermilab (CDF) \cite{Acosta:2004wqa} \cite{Aaltonen:2010rm} and CMS \cite{Khachatryan:2010pv}.

%The evidence for MPI comes from high pT events observed in hadron collisions at the ISR at CERN [1] and later
%at the Fermilab Tevatron collider[2, 3, 4]. At lower pT, underlying event (UE) observables have been measured in pp
%collisions in dijet and Drell-Yan events at CDF in Run I [5] and Run II [6] at centre-of-mass energies of p
%s = 1:8 TeV
%and 1:96 TeV respectively, and in pp collisions at p
%s = 900 GeV in a detector-specic study by CMS [7].
%At small transverse momentum MPI have been shown to be necessary for the successful description of th

It is important to understand MPI for several reasons. In the case of rare and exotic physics, two simultaneous non-exotic hard interactions may lead to non-standard looking events, it follows that in order to identify rare interactions and claim a discovery, one must have some understanding of the effects from MPI. Furthermore understanding MPI gives greater insight into the physics of the proton; one of the most stable bound states and fundamental constituents of ordinary matter.

MPI models are generally dependent on modelling the impact parameter between incident protons - the minimum radial distance between their centroid trajectories. This gives the amount of overlap between the effective cross-sectional area of the protons, the smaller the impact parameter i.e. the greater the overlap, the greater the probability of interaction and thus multiple interactions. 

% Sources
% - http://arxiv.org/abs/1111.0469, Multi-Parton Interactions at the LHC, Nov 2011