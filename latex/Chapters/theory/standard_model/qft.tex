\subsection{Quantum Field Theory (QFT)}

Quantum field theory is built on concepts from Quantum Mechanics, Special Relativity and Classical Field Theory. Fundamental particles are described as excitations or quanta of the fields. For example, electrons are quanta of the electron field and similarly photons are quanta of the electromagnetic field i.e. interactions between electrons can be described as being a result of the interaction between the electron field and electromagnetic field. Mathematically these interactions can be described using the Lagrangian Formalism. 

\subsubsection{Lagrangian Formalism}
With Lagrangian mechanics the equations of motion for a given field is derived by minimising the action $\mathcal{S}$ given by,

\begin{equation}
	\mathcal{S} = \int\mathcal{L}(\phi, \partial_\mu \phi) \mathrm{d}^4x
\end{equation}

where $\mathcal{L}$ is the Lagrangian density, $\phi$ is the field and $\partial_\mu$ is the differential operator acting on the space and time coordinates of the field as seen in special relativity. By applying the condition of the Principle of Least Action, the equations of motion are given by the Euler-Lagrange equation,

\begin{equation}
	\partial_\mu(\frac{\partial\mathcal{L}}{\partial(\partial_\mu\phi_i)}) = \frac{\partial\mathcal{L}}{\partial\phi_i}
\end{equation}

\subsection{Gauge Theories}
\label{section: gauge theories}

A gauge theory is defined by a Lagrangian which is invariant under continuous local transformations of the fields or coordinates. 

\begin{equation}
	\delta \mathcal{L} = 0
\end{equation}

Each possible gauge transformations can be represented by a matrix; together these matrices form a group under matrix multiplication - the symmetry group of the gauge theory.

For each generator of the group there is an associated gauge field, for example, in QED there is one generator to the U(1) group which is associated to the electromagnetic four-vector potential field. Similarly, in QCD there are 8 generators associated to the SU(3) group corresponding to 8 gluon fields. The quanta of the gauge fields are called gauge bosons, for the previous examples these are the photon and gluons respectively.

The symmetry group for the Standard Model is U1 x SU(2) x SU(3), it is a non-Abelian group with 12 gauge fields; the corresponding gauge bosons are the photon, W+, W-, Z0 and eight types of gluon.

%\subsection{Coupling Constants}

The coupling constants of a theory are dimensionless values that describe the strength of an interaction. For example the fine structure constant of QED ($\alpha$) describes the strength of the electromagnetic interaction, defined as,

\begin{equation}
	\alpha = \frac{e^2}{4\pi}
	\label{equation: fine structure constant}
\end{equation}

where $e$ is the charge of the positron\footnote{Expressed in Heaviside-Lorentz and natural units. Unless explicitly stated otherwise all following equations will be expressed in this way} and $\alpha$ has the value $1/137$. Theories with coupling constants that have a value much less than one are said to be weakly coupled. The evolution of systems described by these theories are compatible with perturbative calculations in which the expansion is based on powers of the coupling constant. Conversely theories with coupling constants that have a value of the order of one or greater are said to be strongly coupled and are not compatible with the perturbation method.

The Standard model consists of theories with running coupling constants, which vary depending on the energy scale of a process. The behaviour of these are described by the $\beta$ functions,

\begin{equation}
	\beta(g) = \frac{\partial{g}}{\partial{log(\mu)}}
\end{equation}

where g is the coupling constant of the theory ($g = e$ for QED) and $\mu$ is the interaction energy scale. A $\beta$ function with positive values describes a coupling that increases with the energy of the process and vice-versa. 
\subsection{Coupling Constants}

The coupling constants of a theory are dimensionless values that describe the strength of an interaction. For example the fine structure constant of QED ($\alpha$) describes the strength of the electromagnetic interaction, defined as,

\begin{equation}
	\alpha = \frac{e^2}{4\pi}
	\label{equation: fine structure constant}
\end{equation}

where $e$ is the charge of the positron\footnote{Expressed in Heaviside-Lorentz and natural units. Unless explicitly stated otherwise all following equations will be expressed in this way} and $\alpha$ has the value $1/137$. Theories with coupling constants that have a value much less than one are said to be weakly coupled. The evolution of systems described by these theories are compatible with perturbative calculations in which the expansion is based on powers of the coupling constant. Conversely theories with coupling constants that have a value of the order of one or greater are said to be strongly coupled and are not compatible with the perturbation method.

The Standard model consists of theories with running coupling constants, which vary depending on the energy scale of a process. The behaviour of these are described by the $\beta$ functions,

\begin{equation}
	\beta(g) = \frac{\partial{g}}{\partial{log(\mu)}}
\end{equation}

where g is the coupling constant of the theory ($g = e$ for QED) and $\mu$ is the interaction energy scale. A $\beta$ function with positive values describes a coupling that increases with the energy of the process and vice-versa. 
\subsection{Quantum Electrodynamics (QED)}

QED is an example of a Quantum Field Theory, it describes the electromagnetic interactions between charged fermions via the exchange of photons - gauge bosons of the theory. It is both a Quantum Field Theory as well as a Gauge Theory with a symmetry group of U(1) - an Abelian group of composed of 1 x 1 unitary matrices. The Electroweak theory of the standard model is a unification of QED and Quantum Flavour Dynamics - a gauge theory which describes the weak interaction. The Lagrangian for the QED is given by,

\begin{equation}
	\mathcal{L} = \bar{\psi} (i\gamma^\mu D_\mu - m)\psi - \frac{1}{4}F_{\mu\nu}F^{\mu\nu}
\end{equation}

where $\psi$ is a bispinor field of spin $1/2$ corresponding to the electron field; $\gamma^\mu$ are the Dirac Matrices; $\bar{\psi}$ is the Dirac adjoint spinor $\psi^\dagger \gamma^0$; $D_\mu$ is the gauge covariant derivative given by,

\begin{equation}
	D_\mu = \partial_\mu + ieA_\mu + ieB_\mu
\end{equation}

$e$ is the coupling constant between the electron and electromagnetic fields - charge of an electron; $A_\mu$ is the covariant four-potential of the electromagnetic field generated by the electron; $B_\mu$ is the external field due to an external source and $F_{\mu\nu}$ is the electromagnetic field tensor given by,

\begin{equation}
	F_{\mu\nu} = \partial_\mu A_\nu - \partial_\nu A_\mu
\end{equation}
