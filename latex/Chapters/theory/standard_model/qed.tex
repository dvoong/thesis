\subsection{Quantum Electrodynamics (QED)}

QED is an example of a Quantum Field Theory, it describes the electromagnetic interactions between charged fermions via the exchange of photons - gauge bosons of the theory. It is both a Quantum Field Theory as well as a Gauge Theory with a symmetry group of U(1) - an Abelian group of composed of 1 x 1 unitary matrices. The Electroweak theory of the standard model is a unification of QED and Quantum Flavour Dynamics - a gauge theory which describes the weak interaction. The Lagrangian for the QED is given by,

\begin{equation}
	\mathcal{L} = \bar{\psi} (i\gamma^\mu D_\mu - m)\psi - \frac{1}{4}F_{\mu\nu}F^{\mu\nu}
\end{equation}

where $\psi$ is a bispinor field of spin $1/2$ corresponding to the electron field; $\gamma^\mu$ are the Dirac Matrices; $\bar{\psi}$ is the Dirac adjoint spinor $\psi^\dagger \gamma^0$; $D_\mu$ is the gauge covariant derivative given by,

\begin{equation}
	D_\mu = \partial_\mu + ieA_\mu + ieB_\mu
\end{equation}

$e$ is the coupling constant between the electron and electromagnetic fields - charge of an electron; $A_\mu$ is the covariant four-potential of the electromagnetic field generated by the electron; $B_\mu$ is the external field due to an external source and $F_{\mu\nu}$ is the electromagnetic field tensor given by,

\begin{equation}
	F_{\mu\nu} = \partial_\mu A_\nu - \partial_\nu A_\mu
\end{equation}
