%\chapter{Conclusion} % Write in your own chapter title
%\label{Conclusion}
\chapter{Conclusion}
\lhead{\emph{Conclusion}} % Write in your own chapter title to set the page header

Outlined in this thesis are the key physics underlying particle production phenomena as well as prospective areas of interest in this field of physics. The LHCb detector is highly suited to this are of research due especially to its exceptional tracking as well as its overall performance in all areas. 

A method in which the effective position of Hybrid Photo-Detectors (HPDs) in Cherenkov imaging detectors may be aligned by software was presented. This has shown to have a significant increase in the ability of the detector to distinguish particle species, aiding in many of the key measurements made at the LHCb detector. The particle identification power of the LHCb detector also gives a unique opportunity of studying the charged particle production of individual particle species further constraining current event generator models.

Comparisons between the unfolded charged particle distributions and Monte Carlo event generated distributions show significant differences between measured data and MC simulators (as well as between the different event generators). The measurements in this thesis will hopefully go on to help constrain these generators, increasing their predictive powers well as providing insight into the phenomena of particle production in the soft QCD regime. Understanding these phenomena is particularly important for physics at the LHC since multiplicity sensitive phenomena such additional hard or semi-hard scatters are predicted to be more prevalent at the collision energies present at the LHC \cite{arXiv:1111.0469}.

With the increasing collision energies at luminosity at the LHC, the future at the LHC promises to be an exciting and challenging environment in which great advances will be made.