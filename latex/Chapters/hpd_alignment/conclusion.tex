%%%%%%%%%%%%%%%%%%%%%%%%%%%%%%%%%%%%%%%%%%%%%%%%%%%%%%%%%%%%%%%%%%%%%%%%
%                                                                                   		CONCLUSION													        	        %
%%%%%%%%%%%%%%%%%%%%%%%%%%%%%%%%%%%%%%%%%%%%%%%%%%%%%%%%%%%%%%%%%%%%%%%%
\section{Conclusion}
The 2011 alignment software shows significant improvements over the software used in the 2010 alignment. The general fitting stability, accuracy and run dependence monitoring are amongst the key improvements. Over runs from 2010 the improvement in Cherenkov angle resolution is $\sim8\%$ in RICH1 and  $\sim7\%$ in RICH2 and has brought the average Cherenkov angle resolution much close to the expected Monte Carlo average (see table \ref{tab:mean Cherenkov angle resolution 2010 data}). \newline

\begin{table}[htdp]
	\begin{center}
		\caption{Mean Cherenkov angle resolution for 2010 data, in mrad}
		\begin{tabular}{|c|c|c|c|}
			\hline
			RICH & $<\mu>$ (from MC) & $\mu$ (2010 software) & $\mu$ (2011 software) \\
			\hline
			1 & 1.55 & 1.75 & 1.62 \\
			2 & 0.68 & 0.73 & 0.68 \\
			\hline
		\end{tabular}
		\label{tab:mean Cherenkov angle resolution 2010 data}
	\end{center}
\end{table}