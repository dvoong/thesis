\subsection{Background Corrected Distributions}
\label{subsection: charged particle density, background corrected distributions}

%Effects that involve particles being undetected are group together under efficiency effects. There are many reasons why this can occur such as from particles which travel along the beam pipe of an accelerator bypassing the sensitive instruments of the detector; particles which leave weak signals in sensitive components; have low momentum and are swept out of the detector; neutral particles which interact weakly etc. The result of these effects are to decrease the ratio of number of detected particles to the number of generated particles. Conversely effects which produce false, duplicate copies or detector induced particles are grouped together as false or background particles. This may occur when hits from several particles incorrectly associated together; hits from a particle are not all associated together but instead form subgroups; particles interact with material in the detector to cause showers of particles. In order to understand the underlying physics of the initial interaction these effects must be disentangled, this process is known as unfolding.
%
%The first stage of the unfolding procedure is to remove the background contribution from the kinematic track distributions and track multiplicity distributions. Background tracks are classified as follows,

The main sources of the background contributions are shown in table \ref{table: background track classifications}.


%Background contributions result in an increase in the charged particle multiplicity in a given region. To correct the multiplicity in a region a purity weight is applied to each region such that if the fraction of tracks originating from background is small then the corresponding purity is large. 
%
%
%MC studies show that the largest source of background is from the reconstruction of ghost tracks

\begin{table}[htdp]
	\caption{Background track classifications}
		\begin{center}
			\begin{tabular}{|c|p{0.7\textwidth}|}
				\hline
				Ghost track & Tracks which are either a) not associated to any corresponding true particle either from the initial proton-proton interaction or from their interactions with detector material and magnetic fields b) has only a small fraction of hits associated from a true particle.  \\
				\hline
				 Material track & Tracks which are a result of the interaction between particles from the initial proton-proton reaction and detector material. In MC simulated data material tracks are determined by tracks which are associated to particles that are not involved in the generator level event description \\
				 \hline
				 Secondary track & Tracks associated to true particles which do not meet the prompt particle requirement \\
				 \hline
				 Clone track & Tracks associated to true prompt particles and share the same association with other tracks in the event \\
				\hline
			\end{tabular}
		\end{center}
	\label{table: background track classifications}
\end{table}%

To correct for background contributions the mean purity, $p$ is calculated as a function of $\eta$, $p_\mathrm{T}$, $n_\mathrm{VELO}$, $n_\mathrm{t}$ , where $n_\mathrm{VELO}$ is the number of VELO tracks (Tracks that are reconstructed from only hits in the VELO sub-detector, see section \ref{section: velo}) reconstructed in the event and $n_\mathrm{t}$ is the number of hits in the T-stations (section \ref{section: tracking}) for the event. The purity is calculated from MC data using truth information to determine signal tracks. The purity is given by,

\begin{equation}
	p(\eta, p_\mathrm{T}, n_\mathrm{VELO}, n_\mathrm{t}) = \frac{n_\mathrm{mat}(\eta, p_\mathrm{T}, n_\mathrm{VELO}, n_\mathrm{t})}{n_\mathrm{reco}(\eta, p_T, n_\mathrm{VELO}, n_\mathrm{t})}
\end{equation}

where $n_\mathrm{mat}$ is the number of tracks matched to a prompt particle in the same $\eta$, $p_T$ and $n_{velo}$ bin. The background corrected $\eta$ and $p_T$ distributions are calculated by weighting each track by the purity corresponding to the $\eta$ and $p_T$ bin it is in as well as the $n_{velo}$ and $n_\mathrm{t}$ bin for the event. Figure \ref{fig: signal weights} shows the purity as a function of pseudorapidity and transverse momentum.

\begin{figure}
	\centering
	\begin{subfigure}{0.49\textwidth}
		\includegraphics[width=\textwidth]{/afs/cern.ch/work/d/dvoong/private/cmtuser/DaVinci_v33r6/Phys/ChargedParticleMultiplicity/python/efficiency_and_purity/data_files/bk/Down/mc/-1/-1/bk/Down/mc/-1/-1/meissner/pngs/purity/signal.png}
		\caption{Signal - a prompt particle reconstructed in any $\eta$/$p_\mathrm{T}$ bin in the event}
		\label{fig: signal weight}
	\end{subfigure}
	\begin{subfigure}{0.49\textwidth}
		\includegraphics[width=\textwidth]{/afs/cern.ch/work/d/dvoong/private/cmtuser/DaVinci_v33r6/Phys/ChargedParticleMultiplicity/python/efficiency_and_purity/data_files/bk/Down/mc/-1/-1/bk/Down/mc/-1/-1/meissner/pngs/purity/signal_matched.png}
		\caption{Signal - a prompt particle reconstructed in the same bin as its associated track}
		\label{fig: signal matched weight}
	\end{subfigure}
%	\begin{subfigure}{0.49\textwidth}
%		\includegraphics[width=\textwidth]{/afs/cern.ch/work/d/dvoong/private/cmtuser/DaVinci_v33r6/Phys/ChargedParticleMultiplicity/python/efficiency_and_purity/data_files/bk/Down/mc/-1/-1/bk/Down/mc/-1/-1/meissner/pngs/purity/outside_bin.png}
%		\caption{Bin Migration Rate - Contributions from tracks that are associated to prompt particles in a different $\eta$ or $\pt$ bin}
%		\label{}
%	\end{subfigure}
	\caption{Signal weights as a function of $\eta$ and $\mathrm{p}_T$.}
	\label{fig: signal weights}
\end{figure}

Similarly the background rates can be calculated for each of the sources of background in order to gauge the main sources of background, see figure \ref{fig: background rates}. The largest source of background is from ghost tracks with a background rate of the order of 10\% whilst the contributions from secondary and material tracks are of the order of 4\% , finally the contributions from clones are the smallest and are of the order of 0.1\%. The clone tracks are predominantly localised in the region $3.5 \le \eta \le 4.0$; analysis of MC data suggests this is due to the geometry of the VELO detector. In this pseudorapidity range it is highly probably that a particle will pass through two separated sets of modules (see figure \ref{fig: velo angular acceptance}). Two tracks may be reconstructed, one from the hits in the first stations and one from the hits in the second station - producing two clone tracks. Outside of this range the probability of clone tracks being reconstructed rapidly drops as shown in figure \ref{fig: clone rate}.

\begin{figure}
	\centering
	\begin{subfigure}[h]{0.49\textwidth}
		\includegraphics[width=\textwidth]{/afs/cern.ch/work/d/dvoong/private/cmtuser/DaVinci_v33r6/Phys/ChargedParticleMultiplicity/python/efficiency_and_purity/data_files/bk/Down/mc/-1/-1/bk/Down/mc/-1/-1/meissner/pngs/purity/ghost.png}
		\caption{Ghost Rate}
		\label{}
	\end{subfigure}
	\begin{subfigure}[h]{0.49\textwidth}
		\includegraphics[width=\textwidth]{/afs/cern.ch/work/d/dvoong/private/cmtuser/DaVinci_v33r6/Phys/ChargedParticleMultiplicity/python/efficiency_and_purity/data_files/bk/Down/mc/-1/-1/bk/Down/mc/-1/-1/meissner/pngs/purity/secondary.png}
		\caption{Secondary Rate}
		\label{}
	\end{subfigure}
	\begin{subfigure}[h]{0.49\textwidth}
		\includegraphics[width=\textwidth]{/afs/cern.ch/work/d/dvoong/private/cmtuser/DaVinci_v33r6/Phys/ChargedParticleMultiplicity/python/efficiency_and_purity/data_files/bk/Down/mc/-1/-1/bk/Down/mc/-1/-1/meissner/pngs/purity/material.png}
		\caption{Material Rate}
		\label{}
	\end{subfigure}
	\begin{subfigure}[h]{0.49\textwidth}
		\includegraphics[width=\textwidth]{/afs/cern.ch/work/d/dvoong/private/cmtuser/DaVinci_v33r6/Phys/ChargedParticleMultiplicity/python/efficiency_and_purity/data_files/bk/Down/mc/-1/-1/bk/Down/mc/-1/-1/meissner/pngs/purity/clone.png}
		\caption{Clone Rate}
		\label{fig: clone rate}
	\end{subfigure}
	\caption{Background Rates}
	\label{fig: background rates}
\end{figure}

The background corrected distributions are shown in figure \ref{fig: background corrected track distributions}. A comparison between the background corrected distributions in measured data and MC data is shown in figure \ref{fig: background corrected track distributions, MC and measured data comparison}. The dip shown in the pseudorapidity range $4.0 \le \eta \le 4.3$ is due to the flange \cite{Alves:1129809} in the LHCb detector, this presents additional non-sensitive detector material, decreasing the reconstruction efficiency in this region. %A cross check was applied to the MC data to validate the correction method, the results are shown in figure \ref{fig: background correction cross-check}

\begin{figure}[h]
	\begin{subfigure}{0.49\textwidth}
		\includegraphics[width=\textwidth]{/afs/cern.ch/user/d/dvoong/cmtuser/DaVinci_v33r6/Phys/ChargedParticleMultiplicity/python/kinematic_distributions/tracks/data_files/plots/bk/Down/mc/-1/-1/bk/Down/mc/-1/-1/meissner/bk/Down/real/-1/-1/bk/Down/real/-1/-1/pngs/track_distributions/background_corrected/eta_norm_event.png}
		\caption{$\eta$}
		\label{fig: background corrected track distributions eta}
	\end{subfigure}
	\begin{subfigure}[h]{0.49\textwidth}
		\includegraphics[width=\textwidth]{/afs/cern.ch/user/d/dvoong/cmtuser/DaVinci_v33r6/Phys/ChargedParticleMultiplicity/python/kinematic_distributions/tracks/data_files/plots/bk/Down/mc/-1/-1/bk/Down/mc/-1/-1/meissner/bk/Down/real/-1/-1/bk/Down/real/-1/-1/pngs/track_distributions/background_corrected/pt_norm_event.png}
		\caption{$p_T$}
		\label{fig: background corrected track distributions pt}
	\end{subfigure}
	\caption{Background corrected track distributions}
	\label{fig: background corrected track distributions}
\end{figure}

\begin{figure}[h]
	\begin{subfigure}{0.49\textwidth}
		\includegraphics[width=\textwidth]{/afs/cern.ch/user/d/dvoong/cmtuser/DaVinci_v33r6/Phys/ChargedParticleMultiplicity/python/kinematic_distributions/tracks/data_files/plots/bk/Down/mc/-1/-1/bk/Down/mc/-1/-1/meissner/bk/Down/real/-1/-1/bk/Down/real/-1/-1/pngs/comparison/background_corrected/eta_comparison_norm_event.png}
		\caption{$\eta$}
		\label{fig: background corrected track distributions eta}
	\end{subfigure}
	\begin{subfigure}{0.49\textwidth}
		\includegraphics[width=\textwidth]{/afs/cern.ch/user/d/dvoong/cmtuser/DaVinci_v33r6/Phys/ChargedParticleMultiplicity/python/kinematic_distributions/tracks/data_files/plots/bk/Down/mc/-1/-1/bk/Down/mc/-1/-1/meissner/bk/Down/real/-1/-1/bk/Down/real/-1/-1/pngs/comparison/background_corrected/pt_comparison_norm_event.png}
		\caption{$p_T$}
		\label{fig: background corrected track distributions pt}
	\end{subfigure}
	\caption{Background corrected track distributions, MC (blue) and measured data (black) comparison}
	\label{fig: background corrected track distributions, MC and measured data comparison}
\end{figure}

%\begin{figure}[h]
%	\begin{subfigure}{0.49\textwidth}
%		\includegraphics[width=\textwidth]{/afs/cern.ch/user/d/dvoong/cmtuser/DaVinci_v33r6/Phys/ChargedParticleMultiplicity/python/kinematic_distributions/tracks/data_files/plots/bk/Down/mc/-1/-1/bk/Down/mc/-1/-1/meissner/bk/Down/mc/-1/-1/bk/Down/mc/-1/-1/pngs/cross_check/eta_background_correction_cross_check.png}
%		\caption{$\eta$}
%	\end{subfigure}
%	\begin{subfigure}{0.49\textwidth}
%		\includegraphics[width=\textwidth]{/afs/cern.ch/user/d/dvoong/cmtuser/DaVinci_v33r6/Phys/ChargedParticleMultiplicity/python/kinematic_distributions/tracks/data_files/plots/bk/Down/mc/-1/-1/bk/Down/mc/-1/-1/meissner/bk/Down/mc/-1/-1/bk/Down/mc/-1/-1/pngs/cross_check/pt_background_correction_cross_check.png}
%		\caption{$p_T$}
%	\end{subfigure}
%	\caption{Background correction cross-check. Background corrected track distributions are compared with tracks matched to generator prompt particles by MC truth matching}
%	\label{fig: background correction cross-check}
%\end{figure}