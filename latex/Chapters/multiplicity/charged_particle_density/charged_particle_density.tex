\section{Charged Particle Density}
\label{section: charged particle density}

The charged particle density is investigated as a function of pseudorapidity and transverse momentum. The uncorrected distributions are shown in figure \ref{fig: reconstructed eta mag down} and \ref{fig: reconstructed pt mag down} for pseudorapidity and transverse momentum respectively. Comparisons between measured data and MC data are shown in figure \ref{fig: reconstructed track eta, phi, pt and p mag down comparison}.

\begin{figure}[h]
	\begin{subfigure}[h]{0.49\textwidth}
		\includegraphics[width=\textwidth]{/afs/cern.ch/user/d/dvoong/cmtuser/DaVinci_v33r6/Phys/ChargedParticleMultiplicity/python/kinematic_distributions/tracks/data_files/plots/bk/Down/mc/-1/-1/bk/Down/mc/-1/-1/meissner/bk/Down/real/-1/-1/bk/Down/real/-1/-1/pngs/track_distributions/eta_norm_event.png}
		\caption{$\eta$}
		\label{fig: reconstructed eta mag down}
	\end{subfigure}
	\centering
	\begin{subfigure}[h]{0.49\textwidth}
		\includegraphics[width=\textwidth]{/afs/cern.ch/user/d/dvoong/cmtuser/DaVinci_v33r6/Phys/ChargedParticleMultiplicity/python/kinematic_distributions/tracks/data_files/plots/bk/Down/mc/-1/-1/bk/Down/mc/-1/-1/meissner/bk/Down/real/-1/-1/bk/Down/real/-1/-1/pngs/track_distributions/pt_norm_event.png}
		\caption{$p_T$ (MeV)}
		\label{fig: reconstructed pt mag down}
	\end{subfigure}
	\caption{Uncorrected reconstructed track $\eta$ and $p_T$ of magnet down data.}
	\label{fig: reconstructed track eta, phi, pt and p mag down}
\end{figure}

\begin{figure}
	\begin{subfigure}[h]{0.49\textwidth}
		\includegraphics[width=\textwidth]{/afs/cern.ch/user/d/dvoong/cmtuser/DaVinci_v33r6/Phys/ChargedParticleMultiplicity/python/kinematic_distributions/tracks/data_files/plots/bk/Down/mc/-1/-1/bk/Down/mc/-1/-1/meissner/bk/Down/real/-1/-1/bk/Down/real/-1/-1/pngs/comparison/eta_comparison_norm_event.png}
		\caption{$\eta$}
		\label{fig: reconstructed eta mag down comparison}
	\end{subfigure}
	\centering
	\begin{subfigure}[h]{0.49\textwidth}
		\includegraphics[width=\textwidth]{/afs/cern.ch/user/d/dvoong/cmtuser/DaVinci_v33r6/Phys/ChargedParticleMultiplicity/python/kinematic_distributions/tracks/data_files/plots/bk/Down/mc/-1/-1/bk/Down/mc/-1/-1/meissner/bk/Down/real/-1/-1/bk/Down/real/-1/-1/pngs/comparison/pt_comparison_norm_event.png}
		\caption{$p_T$ (MeV)}
		\label{fig: reconstructed pt mag down comparison}
	\end{subfigure}
	\caption{Uncorrected reconstructed track $\eta$ and $p_T$ of magnet down data. MC data is shown in blue.}
	\label{fig: reconstructed track eta, phi, pt and p mag down comparison}
\end{figure}

The true distributions are obscured by detector effects such as detection inefficiencies or the reconstruction of fake tracks. In order to make a measurement of the true distribution several correction procedures are applied. Firstly a background correction is applied to remove the contributions from tracks that are not associated to any true particle but are instead due to mis-reconstruction effects, secondly an efficiency correction is applied which corrects for prompt particles that are not reconstructed i.e. not observed by the detector. This may be due to charged particles being bent outside of the detector by the magnetic field therefore not leaving any trace in the sub-detectors downstream of the magnet, particles that do not induce enough of a response from the detector to be reconstructed or particles that traverse non-sensitive components of the detector. Lastly a pile-up correction is made in order to remove the contribution from events where there are multiple proton-proton interactions i.e. giving a measurement of the charged particle density for single proton-proton interaction events only.

% (the effect can bee seen in the measured/MC data comparisons figures \ref{fig: reconstructed eta mag down comparison} and \ref{fig: reconstructed pt mag down comparison}
% that corrects for the number of particles observed depending as a function of the region in pseudorapidity and transverse momentum where the multiplicity measurement is made, 
% due to the exclusion of pile-up effects in MC data)

%\subsection{Uncorrected Distributions}
\label{subsection: charged particle density, uncorrected distributions}
\subsection{Background Correction}
\label{subsection: charged particle multiplicity, background correction}

%\begin{figure}[h]
%	\centering
%	\begin{subfigure}{0.32\textwidth}
%		\includegraphics[width=\textwidth]{/afs/cern.ch/user/d/dvoong/cmtuser/DaVinci_v33r6/Phys/ChargedParticleMultiplicity/python/multiplicity/tracks/data_files/TrackMultiplicityPlottingJob/bk/Down/mc/-1/-1/bk/Down/mc/-1/-1/meissner_multiplicity_full/bk/Down/real/-1/-1/bk/Down/real/-1/-1/pngs/background_corrected/2-0_4-5_norm.png}
%		\caption{$2.0 \le \eta \le 4.5$}
%	\end{subfigure}
%	\begin{subfigure}{0.32\textwidth}
%		\includegraphics[width=\textwidth]{/afs/cern.ch/user/d/dvoong/cmtuser/DaVinci_v33r6/Phys/ChargedParticleMultiplicity/python/multiplicity/tracks/data_files/TrackMultiplicityPlottingJob/bk/Down/mc/-1/-1/bk/Down/mc/-1/-1/meissner_multiplicity/bk/Down/real/-1/-1/bk/Down/real/-1/-1/pngs/background_corrected/2-0_2-5_norm.png}
%		\caption{$2.0 \le \eta \le 2.5$}
%	\end{subfigure}
%	\begin{subfigure}{0.32\textwidth}
%		\includegraphics[width=\textwidth]{/afs/cern.ch/user/d/dvoong/cmtuser/DaVinci_v33r6/Phys/ChargedParticleMultiplicity/python/multiplicity/tracks/data_files/TrackMultiplicityPlottingJob/bk/Down/mc/-1/-1/bk/Down/mc/-1/-1/meissner_multiplicity/bk/Down/real/-1/-1/bk/Down/real/-1/-1/pngs/background_corrected/2-5_3-0_norm.png}
%		\caption{$2.5 \le \eta \le 3.0$}
%	\end{subfigure}
%	\begin{subfigure}{0.32\textwidth}
%		\includegraphics[width=\textwidth]{/afs/cern.ch/user/d/dvoong/cmtuser/DaVinci_v33r6/Phys/ChargedParticleMultiplicity/python/multiplicity/tracks/data_files/TrackMultiplicityPlottingJob/bk/Down/mc/-1/-1/bk/Down/mc/-1/-1/meissner_multiplicity/bk/Down/real/-1/-1/bk/Down/real/-1/-1/pngs/background_corrected/3-0_3-5_norm.png}
%		\caption{$3.0 \le \eta \le 3.5$}
%	\end{subfigure}
%	\begin{subfigure}{0.32\textwidth}
%		\includegraphics[width=\textwidth]{/afs/cern.ch/user/d/dvoong/cmtuser/DaVinci_v33r6/Phys/ChargedParticleMultiplicity/python/multiplicity/tracks/data_files/TrackMultiplicityPlottingJob/bk/Down/mc/-1/-1/bk/Down/mc/-1/-1/meissner_multiplicity/bk/Down/real/-1/-1/bk/Down/real/-1/-1/pngs/background_corrected/3-5_4-0_norm.png}
%		\caption{$3.5 \le \eta \le 4.0$}
%	\end{subfigure}
%	\begin{subfigure}{0.32\textwidth}
%		\includegraphics[width=\textwidth]{/afs/cern.ch/user/d/dvoong/cmtuser/DaVinci_v33r6/Phys/ChargedParticleMultiplicity/python/multiplicity/tracks/data_files/TrackMultiplicityPlottingJob/bk/Down/mc/-1/-1/bk/Down/mc/-1/-1/meissner_multiplicity/bk/Down/real/-1/-1/bk/Down/real/-1/-1/pngs/background_corrected/4-0_4-5_norm.png}
%		\caption{$4.0 \le \eta \le 4.5$}
%	\end{subfigure}
%	\caption{Background corrected track multiplicities}
%	\label{fig: background corrected track multiplicities}
%\end{figure}

%To correct for the background contribution to track multiplicity this same method cannot be used since it would result in non-integer multiplicities. Instead the background is modelled with a Poisson distribution,

For the charged particle multiplicity distributions the background is modelled by a Poisson distribution,

\begin{equation*}
	f(k; \lambda) = \frac{\lambda^{k}e^{-\lambda}}{k!}
\end{equation*}

where $k$ corresponds to the number of background tracks in an event and $\lambda$ corresponds to the expected number of background tracks. The expected number of background tracks is calculated by summing the background rates for all tracks in the event. These background rates for the tracks are calculated from the purity calculated in section \ref{subsection: charged particle density, background corrected distributions} and shown in figure \ref{fig: signal weights}. 

\begin{equation}
	\lambda = \sum^{N}_{i=0} 1 - p_i(\eta, p_\mathrm{T}, n_{VELO}, n_\mathrm{t})
\end{equation}

where $N$ is the total number of selected tracks in the event and $p$ is the purity corresponding to the $\eta$, $p_\mathrm{T}$, $n_\mathrm{VELO}$ and $n_\mathrm{t}$ bin associated to the track. To apply the correction to the event multiplicity all allowed values for the number of background tracks ($k$) are considered and weighted by the corresponding probability. An event with $N$ tracks may then be considered as the sum of events with $k \in \{0, 1, ..., N\}$ background tracks weighted by the corresponding probability. Since the Poisson distribution is limited by the allowed values of $k$ ($0 \le k \le N$), the Poisson distribution requires and additional normalisation factor $I^{-1}$ where $I$ is given by, 

\begin{equation}
	I = \sum^{N}_{k=0} f(k; \lambda)
\end{equation}

The results of the background correction applied to measured data are shown in figure \ref{fig: background corrected track multiplicities} and comparisons to the background correction applied to MC data is shown in figure \ref{fig: background corrected track multiplicity comparison}.

%\begin{figure}[h]
%	\centering
%	\begin{subfigure}{0.32\textwidth}
%		\includegraphics[width=\textwidth]{/afs/cern.ch/user/d/dvoong/cmtuser/DaVinci_v33r6/Phys/ChargedParticleMultiplicity/python/multiplicity/tracks/data_files/TrackMultiplicityPlottingJob/bk/Down/mc/-1/-1/bk/Down/mc/-1/-1/meissner_multiplicity_full/bk/Down/real/-1/-1/bk/Down/real/-1/-1/pngs/comparison/background_corrected/2-0_4-5_comparison.png}
%		\caption{$2.0 \le \eta \le 4.5$}
%	\end{subfigure}
%	\begin{subfigure}{0.32\textwidth}
%		\includegraphics[width=\textwidth]{/afs/cern.ch/user/d/dvoong/cmtuser/DaVinci_v33r6/Phys/ChargedParticleMultiplicity/python/multiplicity/tracks/data_files/TrackMultiplicityPlottingJob/bk/Down/mc/-1/-1/bk/Down/mc/-1/-1/meissner_multiplicity/bk/Down/real/-1/-1/bk/Down/real/-1/-1/pngs/comparison/background_corrected/2-0_2-5_comparison.png}
%		\caption{$2.0 \le \eta \le 2.5$}
%	\end{subfigure}
%	\begin{subfigure}{0.32\textwidth}
%		\includegraphics[width=\textwidth]{/afs/cern.ch/user/d/dvoong/cmtuser/DaVinci_v33r6/Phys/ChargedParticleMultiplicity/python/multiplicity/tracks/data_files/TrackMultiplicityPlottingJob/bk/Down/mc/-1/-1/bk/Down/mc/-1/-1/meissner_multiplicity/bk/Down/real/-1/-1/bk/Down/real/-1/-1/pngs/comparison/background_corrected/2-5_3-0_comparison.png}
%		\caption{$2.5 \le \eta \le 3.0$}
%	\end{subfigure}
%	\begin{subfigure}{0.32\textwidth}
%		\includegraphics[width=\textwidth]{/afs/cern.ch/user/d/dvoong/cmtuser/DaVinci_v33r6/Phys/ChargedParticleMultiplicity/python/multiplicity/tracks/data_files/TrackMultiplicityPlottingJob/bk/Down/mc/-1/-1/bk/Down/mc/-1/-1/meissner_multiplicity/bk/Down/real/-1/-1/bk/Down/real/-1/-1/pngs/comparison/background_corrected/3-0_3-5_comparison.png}
%		\caption{$3.0 \le \eta \le 3.5$}
%	\end{subfigure}
%	\begin{subfigure}{0.32\textwidth}
%		\includegraphics[width=\textwidth]{/afs/cern.ch/user/d/dvoong/cmtuser/DaVinci_v33r6/Phys/ChargedParticleMultiplicity/python/multiplicity/tracks/data_files/TrackMultiplicityPlottingJob/bk/Down/mc/-1/-1/bk/Down/mc/-1/-1/meissner_multiplicity/bk/Down/real/-1/-1/bk/Down/real/-1/-1/pngs/comparison/background_corrected/3-5_4-0_comparison.png}
%		\caption{$3.5 \le \eta \le 4.0$}
%	\end{subfigure}
%	\begin{subfigure}{0.32\textwidth}
%		\includegraphics[width=\textwidth]{/afs/cern.ch/user/d/dvoong/cmtuser/DaVinci_v33r6/Phys/ChargedParticleMultiplicity/python/multiplicity/tracks/data_files/TrackMultiplicityPlottingJob/bk/Down/mc/-1/-1/bk/Down/mc/-1/-1/meissner_multiplicity/bk/Down/real/-1/-1/bk/Down/real/-1/-1/pngs/comparison/background_corrected/4-0_4-5_comparison.png}
%		\caption{$4.0 \le \eta \le 4.5$}
%	\end{subfigure}
%	\caption{Background corrected track multiplicities from real data and MC}
%	\label{fig: background corrected track multiplicity comparison}
%\end{figure}

\begin{figure}[H]
	\centering
	\begin{subfigure}{0.32\textwidth}
		\includegraphics[width=\textwidth]{Chapters/multiplicity/images/background_corrected/real/2-0_4-5_norm.png}
		\caption{$2.0 \le \eta \le 4.5$}
	\end{subfigure}
	\begin{subfigure}{0.32\textwidth}
		\includegraphics[width=\textwidth]{Chapters/multiplicity/images/background_corrected/real/2-0_2-5_norm.png}
		\caption{$2.0 \le \eta \le 2.5$}
	\end{subfigure}
	\begin{subfigure}{0.32\textwidth}
		\includegraphics[width=\textwidth]{Chapters/multiplicity/images/background_corrected/real/2-5_3-0_norm.png}
		\caption{$2.5 \le \eta \le 3.0$}
	\end{subfigure}
	\begin{subfigure}{0.32\textwidth}
		\includegraphics[width=\textwidth]{Chapters/multiplicity/images/background_corrected/real/3-0_3-5_norm.png}
		\caption{$3.0 \le \eta \le 3.5$}
	\end{subfigure}
	\begin{subfigure}{0.32\textwidth}
		\includegraphics[width=\textwidth]{Chapters/multiplicity/images/background_corrected/real/3-5_4-0_norm.png}
		\caption{$3.5 \le \eta \le 4.0$}
	\end{subfigure}
	\begin{subfigure}{0.32\textwidth}
		\includegraphics[width=\textwidth]{Chapters/multiplicity/images/background_corrected/real/4-0_4-5_norm.png}
		\caption{$4.0 \le \eta \le 4.5$}
	\end{subfigure}
	\caption{Background corrected track multiplicities in measured data}
	\label{fig: background corrected track multiplicities}
\end{figure}

\begin{figure}[H]
	\centering
	\begin{subfigure}{0.32\textwidth}
		\includegraphics[width=\textwidth]{Chapters/multiplicity/charged_particle_event_multiplicity/images/background_correction_comparison/2_0-4_5.png}
		\caption{$2.0 \le \eta \le 4.5$}
	\end{subfigure}
	\begin{subfigure}{0.32\textwidth}
		\includegraphics[width=\textwidth]{Chapters/multiplicity/charged_particle_event_multiplicity/images/background_correction_comparison/2_0-2_5.png}
		\caption{$2.0 \le \eta \le 2.5$}
	\end{subfigure}
	\begin{subfigure}{0.32\textwidth}
		\includegraphics[width=\textwidth]{Chapters/multiplicity/charged_particle_event_multiplicity/images/background_correction_comparison/2_5-3_0.png}
		\caption{$2.5 \le \eta \le 3.0$}
	\end{subfigure}
	\begin{subfigure}{0.32\textwidth}
		\includegraphics[width=\textwidth]{Chapters/multiplicity/charged_particle_event_multiplicity/images/background_correction_comparison/3_0-3_5.png}
		\caption{$3.0 \le \eta \le 3.5$}
	\end{subfigure}
	\begin{subfigure}{0.32\textwidth}
		\includegraphics[width=\textwidth]{Chapters/multiplicity/charged_particle_event_multiplicity/images/background_correction_comparison/3_5-4_0.png}
		\caption{$3.5 \le \eta \le 4.0$}
	\end{subfigure}
	\begin{subfigure}{0.32\textwidth}
		\includegraphics[width=\textwidth]{Chapters/multiplicity/charged_particle_event_multiplicity/images/background_correction_comparison/4_0-4_5.png}
		\caption{$4.0 \le \eta \le 4.5$}
	\end{subfigure}
	\caption{Background corrected track multiplicities from measured data and MC}
	\label{fig: background corrected track multiplicity comparison}
\end{figure}

%\begin{figure}[h]
%	\centering
%	\begin{subfigure}{0.32\textwidth}
%		\includegraphics[width=\textwidth]{/afs/cern.ch/user/d/dvoong/cmtuser/DaVinci_v33r6/Phys/ChargedParticleMultiplicity/python/multiplicity/tracks/data_files/TrackMultiplicityPlottingJob/bk/Down/mc/-1/-1/bk/Down/mc/-1/-1/meissner_multiplicity_full/bk/Down/mc/-1/-1/bk/Down/mc/-1/-1/pngs/cross_check/2-0_4-5.png}
%		\caption{$2.0 \le \eta \le 4.5$}
%	\end{subfigure}
%	\begin{subfigure}{0.32\textwidth}
%		\includegraphics[width=\textwidth]{/afs/cern.ch/user/d/dvoong/cmtuser/DaVinci_v33r6/Phys/ChargedParticleMultiplicity/python/multiplicity/tracks/data_files/TrackMultiplicityPlottingJob/bk/Down/mc/-1/-1/bk/Down/mc/-1/-1/meissner_multiplicity/bk/Down/mc/-1/-1/bk/Down/mc/-1/-1/pngs/cross_check/2-0_2-5.png}
%		\caption{$2.0 \le \eta \le 2.5$}
%	\end{subfigure}
%	\begin{subfigure}{0.32\textwidth}
%		\includegraphics[width=\textwidth]{/afs/cern.ch/user/d/dvoong/cmtuser/DaVinci_v33r6/Phys/ChargedParticleMultiplicity/python/multiplicity/tracks/data_files/TrackMultiplicityPlottingJob/bk/Down/mc/-1/-1/bk/Down/mc/-1/-1/meissner_multiplicity/bk/Down/mc/-1/-1/bk/Down/mc/-1/-1/pngs/cross_check/2-5_3-0.png}
%		\caption{$2.5 \le \eta \le 3.0$}
%	\end{subfigure}
%	\begin{subfigure}{0.32\textwidth}
%		\includegraphics[width=\textwidth]{/afs/cern.ch/user/d/dvoong/cmtuser/DaVinci_v33r6/Phys/ChargedParticleMultiplicity/python/multiplicity/tracks/data_files/TrackMultiplicityPlottingJob/bk/Down/mc/-1/-1/bk/Down/mc/-1/-1/meissner_multiplicity/bk/Down/mc/-1/-1/bk/Down/mc/-1/-1/pngs/cross_check/3-0_3-5.png}
%		\caption{$3.0 \le \eta \le 3.5$}
%	\end{subfigure}
%	\begin{subfigure}{0.32\textwidth}
%		\includegraphics[width=\textwidth]{/afs/cern.ch/user/d/dvoong/cmtuser/DaVinci_v33r6/Phys/ChargedParticleMultiplicity/python/multiplicity/tracks/data_files/TrackMultiplicityPlottingJob/bk/Down/mc/-1/-1/bk/Down/mc/-1/-1/meissner_multiplicity/bk/Down/mc/-1/-1/bk/Down/mc/-1/-1/pngs/cross_check/3-5_4-0.png}
%		\caption{$3.5 \le \eta \le 4.0$}
%	\end{subfigure}
%	\begin{subfigure}{0.32\textwidth}
%		\includegraphics[width=\textwidth]{/afs/cern.ch/user/d/dvoong/cmtuser/DaVinci_v33r6/Phys/ChargedParticleMultiplicity/python/multiplicity/tracks/data_files/TrackMultiplicityPlottingJob/bk/Down/mc/-1/-1/bk/Down/mc/-1/-1/meissner_multiplicity/bk/Down/mc/-1/-1/bk/Down/mc/-1/-1/pngs/cross_check/4-0_4-5.png}
%		\caption{$4.0 \le \eta \le 4.5$}
%	\end{subfigure}
%	\caption{Background correction cross-check. Background corrected track multiplicities are compared with tracks matched to generator prompt particles by MC truth matching}
%	\label{fig: background corrected track multiplicity cross-check}
%\end{figure}
\subsection{Efficiency Corrected Distributions}
\label{subsection: charged particle density, efficiency corrected distributions}

The efficiency correction corrects for prompt particles that are not detected by the detector. It is calculated by first modelling the detector response from Monte Carlo data using the truth information to associate tracks to generator level particles. The detector response is independent of the generator model used to create the simulated event so as to not introduce biases. As with the background rate, the efficiency is calculated in bins of pseudorapidity, transverse momentum, VELO track (section \ref{fig: track types}) multiplicity and t-station hit multiplicity. The efficiency in a given bin is given by,

%The efficiency for a given $\eta$ and $p_\mathrm{T}$ bin is the fraction of particles in the bin which are reconstructed as tracks in the same bin; it is the probability for a particle in an $\eta$ and $p_T$ bin to be reconstructed with a reconstructed $\eta$ and $p_T$ in the same bin.

\begin{equation}
	\epsilon(\eta, p_\mathrm{T}, n_\mathrm{VELO}, n_\mathrm{t}) = \frac{n_\mathrm{mat}(\eta, p_\mathrm{T}, n_\mathrm{VELO}, n_t)}{n_\mathrm{true}(\eta, p_\mathrm{T}, n_\mathrm{VELO}, n_\mathrm{t})}
\end{equation}

averaged over all events. Here $n_\mathrm{mat}$ is the number of tracks matched to prompt particles in the same $\eta$ and $p_\mathrm{T}$ bin as before with the purity and $n_\mathrm{true}$ is the number of prompt particles in the $\eta$ and $p_\mathrm{T}$ bin. The efficiency as a function of pseudorapidity and transverse momentum is shown in figure \ref{fig: reconstruction efficiency} averaged over all bins in $n_\mathrm{VELO}$ and $n_\mathrm{t}$.

%\subsection*{$\eta$ and $p_T$ distributions}

\begin{figure}[h]
	\centering
	\begin{subfigure}{0.49\textwidth}
		\includegraphics[width=\textwidth]{/afs/cern.ch/user/d/dvoong/cmtuser/DaVinci_v33r6/Phys/ChargedParticleMultiplicity/python/efficiency_and_purity/data_files/bk/Down/mc/-1/-1/bk/Down/mc/-1/-1/meissner/pngs/efficiency/signal.png}
		\caption{probability of reconstructing a prompt particle as a track anywhere in $\eta$ and $p_\mathrm{T}$}
		\label{fig: reconstruction efficiency, signal}
	\end{subfigure}
	\begin{subfigure}{0.49\textwidth}
		\includegraphics[width=\textwidth]{/afs/cern.ch/user/d/dvoong/cmtuser/DaVinci_v33r6/Phys/ChargedParticleMultiplicity/python/efficiency_and_purity/data_files/bk/Down/mc/-1/-1/bk/Down/mc/-1/-1/meissner/pngs/efficiency/signal_matched.png}
		\caption{probability of reconstructing a prompt particle as a track in the same $\eta$ and $p_\mathrm{T}$ bin}
		\label{fig: reconstruction efficiency, signal matched}
	\end{subfigure}
	\caption{Prompt particle reconstruction efficiency}
	\label{fig: reconstruction efficiency}
\end{figure}

By combining the efficiency together with the purity the average ratio of the number of reconstructed tracks to the number of true prompt particles can be calculated, see equation \ref{equation: correction factor}. This correction factor is then applied on an event by event basis to give the corrected particle density contributions, figure \ref{fig: unfolded track distributions}. A comparison between the corrected charged particle densities of measured and MC data can be seen in figure \ref{fig: unfolded track distributions comparison} and a cross-check of the method using MC data is shown in figure \ref{fig: unfolded track distributions cross-check}.

\begin{equation}
	\frac{n_\mathrm{reco}}{n_\mathrm{true}}(\eta, p_T, n_{VELO}, n_t) = p \cdot \epsilon^{-1}
	\label{equation: correction factor}
\end{equation}

%Corrected $\eta$ and $p_T$ distributions combine the purity and efficiency corrections factors to give the total correction factor, $c$,


\begin{figure}[h]
	\begin{subfigure}[h]{0.49\textwidth}
		\includegraphics[width=\textwidth]{/afs/cern.ch/user/d/dvoong/cmtuser/DaVinci_v33r6/Phys/ChargedParticleMultiplicity/python/kinematic_distributions/tracks/data_files/plots/bk/Down/mc/-1/-1/bk/Down/mc/-1/-1/meissner/bk/Down/real/-1/-1/bk/Down/real/-1/-1/pngs/track_distributions/unfolded/eta_norm_event.png}
		\caption{$\eta$}
		\label{fig: background corrected track distributions eta}
	\end{subfigure}
	\begin{subfigure}[h]{0.49\textwidth}
		\includegraphics[width=\textwidth]{/afs/cern.ch/user/d/dvoong/cmtuser/DaVinci_v33r6/Phys/ChargedParticleMultiplicity/python/kinematic_distributions/tracks/data_files/plots/bk/Down/mc/-1/-1/bk/Down/mc/-1/-1/meissner/bk/Down/real/-1/-1/bk/Down/real/-1/-1/pngs/track_distributions/unfolded/pt_norm_event.png}
		\caption{$p_T$}
		\label{fig: background corrected track distributions pt}
	\end{subfigure}
	\caption{Unfolded track distributions}
	\label{fig: unfolded track distributions}
\end{figure}

\begin{figure}[h]
	\begin{subfigure}[h]{0.49\textwidth}
		\includegraphics[width=\textwidth]{/afs/cern.ch/user/d/dvoong/cmtuser/DaVinci_v33r6/Phys/ChargedParticleMultiplicity/python/kinematic_distributions/tracks/data_files/plots/bk/Down/mc/-1/-1/bk/Down/mc/-1/-1/meissner/bk/Down/real/-1/-1/bk/Down/real/-1/-1/pngs/comparison/unfolded/eta_comparison_norm_event.png}
		\caption{$\eta$}
		\label{fig: background corrected track distributions eta}
	\end{subfigure}
	\begin{subfigure}[h]{0.49\textwidth}
		\includegraphics[width=\textwidth]{/afs/cern.ch/user/d/dvoong/cmtuser/DaVinci_v33r6/Phys/ChargedParticleMultiplicity/python/kinematic_distributions/tracks/data_files/plots/bk/Down/mc/-1/-1/bk/Down/mc/-1/-1/meissner/bk/Down/real/-1/-1/bk/Down/real/-1/-1/pngs/comparison/unfolded/pt_comparison_norm_event.png}
		\caption{$p_T$}
		\label{fig: background corrected track distributions pt}
	\end{subfigure}
	\caption{Unfolded track distributions, measured/MC data comparison}
	\label{fig: unfolded track distributions comparison}
\end{figure}

\begin{figure}[h]
	\begin{subfigure}[h]{0.49\textwidth}
		\includegraphics[width=\textwidth]{/afs/cern.ch/user/d/dvoong/cmtuser/DaVinci_v33r6/Phys/ChargedParticleMultiplicity/python/kinematic_distributions/tracks/data_files/plots/bk/Down/mc/-1/-1/bk/Down/mc/-1/-1/meissner/bk/Down/mc/-1/-1/bk/Down/mc/-1/-1/pngs/cross_check/eta_unfolded_cross_check.png}
		\caption{$\eta$}
		\label{fig: background corrected track distributions eta}
	\end{subfigure}
	\begin{subfigure}[h]{0.49\textwidth}
		\includegraphics[width=\textwidth]{/afs/cern.ch/user/d/dvoong/cmtuser/DaVinci_v33r6/Phys/ChargedParticleMultiplicity/python/kinematic_distributions/tracks/data_files/plots/bk/Down/mc/-1/-1/bk/Down/mc/-1/-1/meissner/bk/Down/mc/-1/-1/bk/Down/mc/-1/-1/pngs/cross_check/pt_unfolded_cross_check.png}
		\caption{$p_T$}
		\label{fig: background corrected track distributions pt}
	\end{subfigure}
	\caption{Unfolded track distributions, MC cross-check}
	\label{fig: unfolded track distributions cross-check}
\end{figure}


\subsection{Pile-Up}
\label{subsection: charged particle event multiplicity, pile-up}

The pile-up contribution to the charged particle event multiplicity is approximated as consisting of the distribution from events with a single proton-proton collision and the convolution of two single proton-proton collisions. The average number of proton-proton interactions is given by equation \ref{equation: mu approximation} for small $\mu$ shown in section \ref{subsection: charged particle density, pile-up}. This means the data is dominated by single proton-proton interactions and may be approximated by,

\begin{equation}
	N_\mathrm{observed}(n) \approx \frac{N_\mathrm{single}(n) + \frac{\mu}{2} N_\mathrm{single}(n) \ast N_\mathrm{single}(n)}{A}
	\label{equation: pile-up multiplicity composition}
\end{equation}

with,

\begin{equation}
	N_\mathrm{single}(n) \ast N_\mathrm{single}(n) = \sum_{k=0}^{n} N_\mathrm{single}(k) \cdot N_\mathrm{single} (n - k)
\end{equation}

$A$ is a normalisation constant given by,

\begin{equation}
	A = \sum_{n=0}^{n_\mathrm{max}} N_\mathrm{single}(n) + \frac{\mu}{2} N_\mathrm{single}(n) \ast N_\mathrm{single}(n)
\end{equation}

 $N_\mathrm{observed}(k)$ is the number of events with $n$ charged particles consisting of contributions of events with one or two proton-proton interactions; $N_\mathrm{single}(k)$ is the number of events with $n$ charged particles from single proton-proton interactions, $k$ is the number of charged particles from a secondary proton-proton interaction in an event with $n$ charged particles and $N_\mathrm{single}(k) \ast N_\mathrm{single} (n - k)$ is the convolution of two single proton-proton interactions producing $n$ charged particles. The normalisation constant $A$ may also be expressed as,

\begin{equation}
	A = 1 + \frac{\mu}{2}
\end{equation}

since $N_\mathrm{single}(n)$ and $N_\mathrm{single}(n) \ast N_\mathrm{single}(n)$ are normalised functions of $n$. Solving equation \ref{equation: pile-up multiplicity composition} for $N_\mathrm{single}$ gives,

\begin{equation}
	N_\mathrm{single}(n) = (1 + \frac{\mu}{2}) N_\mathrm{observed}(n) - \frac{\mu}{2} N_\mathrm{single}(n) \ast N_\mathrm{single}(n)
\end{equation}

The form of this equation suggests an iterative procedure may be used in order to approximate $N_\mathrm{single}$, using this approach gives,

\begin{equation}
	N_\mathrm{single}''(n) = (1 + \frac{\mu}{2})N_\mathrm{observed}(n) - \frac{\mu}{2} N_\mathrm{single}(n)' \ast N_\mathrm{single}(n)'
\end{equation}

where $N_\mathrm{single}'(n)$ is an approximation of $N_\mathrm{single}(n)$ and $N_\mathrm{single}''(n)$ is the next iteration of the approximation of $N_\mathrm{single}(n)$. Starting with $N_\mathrm{observed}(n)$ as the initial seed for the process gives the results shown in figure \ref{fig: n single approximation}. The pile-up correction for high multiplicities are not included due to the poor statistical significance at high multiplicities.

\begin{figure}
	\centering
	\begin{subfigure}{0.49\textwidth}
		\includegraphics[width=\textwidth]{Chapters/multiplicity/charged_particle_event_multiplicity/images/comparison.png}
		\caption{A comparison of the observed event multiplicity to the convolution of the observed event multiplicity}
		\label{fig: convolution}
	\end{subfigure}
	\begin{subfigure}{0.49\textwidth}
		\includegraphics[width=\textwidth]{Chapters/multiplicity/charged_particle_event_multiplicity/images/n_single_approximation_comparison.png}
		\caption{$N_\mathrm{single}$ approximation. The observed distribution is shown in black, the first and second iterations are shown in blue and red respectively.}
		\label{fig: n single approximation}
	\end{subfigure}
	\caption{Pile-up contributions to the charged particle event multiplicity for $2.0 \le \eta \le 4.5$.}
\end{figure}

%Introducing a trigger condition which accepts only events with visible interactions gives the average number of interactions in triggered events i.e. the pile-up ($\mu_1$) as,
%
%\begin{equation}
%	\mu_1 = \frac{\sum^{\infty}_{k=1}{k\cdotp_k}}{\sum^{\infty}_{k=1}{p_k}} = \frac{\sum^{\infty}_{k=0}{k\cdotp_k}}{\sum^{\infty}_{k=1}{p_k}} = \frac{\mu}{1-p_0} = \frac{\mu}{1 - e^{-\mu}}
%\end{equation}
%
%
%
%\begin{equation}
%	\mu_1 \approx \frac{\mu}{1 - (1 - \mu + \mu^2/2)} \approx 1 + \frac{\mu}{2}
%\end{equation}

%At the LHCb detector pile-up is defined as the average number of proton-proton interactions ($n$) in visible events. 

%The number of visible proton-proton interactions ($n$) per bunch crossing follows a Poisson distribution,
%
%\begin{equation}
%	P(n; \mu) = \frac{\mu^n e^{-\mu}}{n!}
%\end{equation}
%
%where $\mu$ is the expected number of visible proton-proton interactions per punch crossing. The pile-up ($\mu_1$) is defined as the 0 suppressed expectation value, for $n \ge 1$, given by (see appendix \ref{AppendixA}),
%
%\begin{equation}
%	\mu_1 = \frac{\mu}{1-e^{-\mu}}
%\end{equation}
%
%$\mu$ is calculated as 0.002 for magnet down data in 2010, see section \ref{subsection: estimation of mu}.
%
%To calculate the multiplicity distribution for events with single proton-proton interactions it is required to correct for effects due to the pile-up. The observed multiplicity distribution ($\theta$) is postulated to be a convolution of single proton-proton interaction distributions ($\phi$), given by,
%
%\begin{equation}
%	\theta(n) = \frac{1}{\sum_{k=1}^{\infty} P(k)} \sum_{k=1}^{\infty}P(k)\phi_k(n)
%\end{equation}
%
%with,
%\begin{equation*}
%	\phi_k = \phi \ast \phi ... \ast \phi \mathrm{\,\,\,\,\,} k \mathrm{\,times}
%\end{equation*}
%
%For small values of $\mu$ it is sufficient to consider only the first two terms. The expression for the pile-up ($\mu_1$) in this regime can be expanded as,
%\begin{equation}
%	\mu_1 \approx \frac{\mu}{1 - (1 - \mu + \mu^2/2)}
%\end{equation}
%or,
%\begin{equation}
%	\mu_1 \approx 1 + \frac{\mu}{2}
%\end{equation}
%
%The expression for the observed distribution in terms of convolutions of single proton-proton interaction distributions approximates to,
%
%\begin{equation}
%	\theta(n) \approx \frac{P(1)\phi(n) + P(2)\phi_2(n)}{P(1) + P(2)} = \frac{\phi(n) + (\mu/2)\phi_2(n)}{1 + \mu/2}
%\end{equation}
%
%In terms of the single proton-proton interaction distribution $\phi$ this is,
%
%\begin{equation}
%	\phi(n) = (1 + \frac{\mu}{2})\theta(n) - \frac{\mu}{2}\phi_2(n)
%\end{equation}
%
%The form of this equation suggests an iterative approach to calculating the solution with,
%
%\begin{equation}
%	\phi^{i+1}(n) = (1 + \frac{\mu}{2})\theta(n) - \frac{\mu}{2}\phi_2^{i}(n) \mathrm{\,\,\,\,\,\,\,\,for\,\,the\,\,} i^{th} \mathrm{\,\,iteration}
%\end{equation}
%
%Since $\mu$ is small a suitable choice at the initial seed for $\phi^0$ is the observed multiplicity $\theta$, this gives the first iteration of $\phi$ as,
%
%\begin{equation}
%	\phi' = (1 + \frac{\mu}{2})\theta - \frac{\mu}{2}\theta_2 = \theta + \frac{\mu}{2}(\theta - \theta_2)
%\end{equation}
%with,
%\begin{equation*}
%	 \theta_k = \theta \ast \theta ... \ast \theta \mathrm{\,\,\,\,\,\,\,\,} k \mathrm{\,\,times}
%\end{equation*}
%
%and for the second iteration,
%
%\begin{equation}
%	\phi'' = \phi' - \frac{\mu^2}{2}(\theta_2 - \theta_3) + \frac{\mu^3}{8}(\theta_2 - 2\theta_3 + \theta_4)
%\end{equation}
%
%The pile-up correction applies only two iteration, the next iteration affects only the $\mu^3$ term which has a small effect on the overall correction due to the smallness of $\mu$.
\subsection{Uncertainties and Results}
\label{subsection: charged particle density, systematics}

\begin{figure}[h]
	\begin{subfigure}{0.49\textwidth}
		\includegraphics[width=\textwidth]{/afs/cern.ch/user/d/dvoong/cmtuser/DaVinci_v33r6/Phys/ChargedParticleMultiplicity/python/kinematic_distributions/tracks/systematics/data_files/TrackDistributionComparisonPlottingJob/bk/Down/real/-1/-1/pngs/eta_background_corrected_overlay.png}
		\caption{Background corrected charged particle density, $\eta$}
	\end{subfigure}
	\begin{subfigure}{0.49\textwidth}
		\includegraphics[width=\textwidth]{/afs/cern.ch/user/d/dvoong/cmtuser/DaVinci_v33r6/Phys/ChargedParticleMultiplicity/python/kinematic_distributions/tracks/systematics/data_files/TrackDistributionComparisonPlottingJob/bk/Down/real/-1/-1/pngs/eta_unfolded_overlay.png}
		\caption{Unfolded charged particle density, $\eta$}
	\end{subfigure}
	\begin{subfigure}{0.49\textwidth}
		\includegraphics[width=\textwidth]{/afs/cern.ch/user/d/dvoong/cmtuser/DaVinci_v33r6/Phys/ChargedParticleMultiplicity/python/kinematic_distributions/tracks/systematics/data_files/TrackDistributionComparisonPlottingJob/bk/Down/real/-1/-1/pngs/pt_background_corrected_overlay.png}
		\caption{Background  charged particle density, $p_\mathrm{T}$}
	\end{subfigure}
	\begin{subfigure}{0.49\textwidth}
		\includegraphics[width=\textwidth]{/afs/cern.ch/user/d/dvoong/cmtuser/DaVinci_v33r6/Phys/ChargedParticleMultiplicity/python/kinematic_distributions/tracks/systematics/data_files/TrackDistributionComparisonPlottingJob/bk/Down/real/-1/-1/pngs/pt_unfolded_overlay.png}
		\caption{Unfolded charged particle density, $p_\mathrm{T}$}
	\end{subfigure}
	\caption{Uncertainty of the corrected charged particle density in measured data due to statistical errors in the purity and efficiency}
	\label{fig: statistical uncertainty of corrected charged particle density, measured data}
\end{figure}

Since the purity and efficiency distributions are calculated from MC generated data, there is an associated uncertainty due to the statistical error in their distributions. This uncertainty is estimated by applying the correction procedure with the values of the purity and efficiency varied by their standard deviation. The resulting corrected distributions are shown in figure \ref{fig: statistical uncertainty of corrected charged particle density, measured data} for measured data and figure \ref{fig: statistical uncertainty of corrected charged particle density, mc data} for MC data.

\begin{figure}[h]
	\begin{subfigure}{0.49\textwidth}
		\includegraphics[width=\textwidth]{/afs/cern.ch/user/d/dvoong/cmtuser/DaVinci_v33r6/Phys/ChargedParticleMultiplicity/python/kinematic_distributions/tracks/systematics/data_files/TrackDistributionComparisonPlottingJob/bk/Down/mc/-1/-1/pngs/eta_background_corrected_overlay.png}
		\caption{Background corrected charged particle density, $\eta$}
	\end{subfigure}
	\begin{subfigure}{0.49\textwidth}
		\includegraphics[width=\textwidth]{/afs/cern.ch/user/d/dvoong/cmtuser/DaVinci_v33r6/Phys/ChargedParticleMultiplicity/python/kinematic_distributions/tracks/systematics/data_files/TrackDistributionComparisonPlottingJob/bk/Down/mc/-1/-1/pngs/eta_unfolded_overlay.png}
		\caption{Unfolded charged particle density, $\eta$}
	\end{subfigure}
	\begin{subfigure}{0.49\textwidth}
		\includegraphics[width=\textwidth]{/afs/cern.ch/user/d/dvoong/cmtuser/DaVinci_v33r6/Phys/ChargedParticleMultiplicity/python/kinematic_distributions/tracks/systematics/data_files/TrackDistributionComparisonPlottingJob/bk/Down/mc/-1/-1/pngs/pt_background_corrected_overlay.png}
		\caption{Background  charged particle density, $p_\mathrm{T}$}
	\end{subfigure}
	\begin{subfigure}{0.49\textwidth}
		\includegraphics[width=\textwidth]{/afs/cern.ch/user/d/dvoong/cmtuser/DaVinci_v33r6/Phys/ChargedParticleMultiplicity/python/kinematic_distributions/tracks/systematics/data_files/TrackDistributionComparisonPlottingJob/bk/Down/mc/-1/-1/pngs/pt_unfolded_overlay.png}
		\caption{Unfolded charged particle density, $p_\mathrm{T}$}
	\end{subfigure}
	\caption{Uncertainty of the corrected charged particle density in MC data due to statistical errors in the purity and efficiency}
	\label{fig: statistical uncertainty of corrected charged particle density, mc data}
\end{figure}

The uncertainty in the background correction of the charged particle density is dominated by the ghost correction, this can been seen in figure \ref{fig: background rates}. The systematic error on the ghost estimation is estimated by comparing the ghost rates calculated in section \ref{subsection: charged particle density, background corrected distributions} to a data driven method of ghost estimation. 

From MC data it can be seen that the dominant source of ghost tracks is from mis-matching VELO track segments (see section \ref{subsection: tracking, track reconstruction}) to hits or track segments in the T-stations - these hits may be due to other particles or detector noise. The data driven method of estimates these effects using the VELO flip method; this involves taking a reconstructed VELO track segment and flipping its direction in x and y. The flipped VELO track segment is then used as a seed in the forward track matching reconstruction algorithm that attempts to match the segment to hits or other track segments in the T-stations. Forward tracks reconstructed using the flipped VELO segment as its seed are then classified as ghost tracks.

In order for the track to be reconstructed it must meet several criteria such as a quality of fit ($\chi^2$) cut. In addition to this the track must be a better candidate than other combinations of the VELO track segment seed and T-station hits or track segments. In order to make an accurate estimation of the ghost rate the track candidates from the flipped VELO track segment are compared to the candidates from the original VELO track segment, if the best track candidate is from the flipped VELO track segment then it is kept.

\begin{figure}[h]
	\centering
	\includegraphics[width=0.49\textwidth]{Chapters/multiplicity/images/ghost_rate_data_driven.png}
	\caption{Ghost rates as a function of T-station hit multiplicity calculated using the data driven VELO flip method. The data points in red correspond to the ghost rate in MC data and the blue data points correspond to measured data. A good agreement is present between MC and measured data.}
	\label{fig: ghost rates, velo flip}
\end{figure}

Due to the nature of the matching algorithm, events with higher activity (expressed by the number of hits in the T-stations) in the detector are expected to have higher ghost rates. The ghost rate is plotted as a function of the T-station multiplicity (figure \ref{fig: ghost rates, velo flip}). The ghost rate is determined by the ratio of ghost classified tracks to the number of VELO track segments in an event, to translate this to the context of long tracks used in this analysis the ghost rate in long tracks is related to the ghost rate in VELO tracks scaled by the ratio of VELO tracks to long tracks. 

\begin{equation}
	R_\mathrm{long} = R_\mathrm{VELO} \cdot \frac{n_\mathrm{VELO}}{n_\mathrm{long}}
\end{equation}

where $R_\mathrm{long}$ is the ghost rate in long tracks, $R_\mathrm{VELO}$ is the ghost rate in VELO tracks, $n_\mathrm{VELO}$ is the number of velo tracks and $n_\mathrm{long}$ is the number of long tracks in the event. The overall ghost rate is then calculated from the average of the ghost rate in all events. The comparison between ghost rate estimation using the VELO flip method is shown in figure \ref{fig: ghost rates, velo flip}; there is good agreement between measured and MC data. A conservative estimate of 2\% for the systematic error was made for this analysis, as is the case for similar analyses such as \cite{Aaij:1662427}.

The final results for the charged particle density are shown in figure \ref{fig: corrected charged particle densities} with the systematics errors shown by the blue shaded boxes. A comparison between the unfolded distribution and charged particle distributions in MC generated events are shown for several event generators in figure \ref{fig: corrected charged particle densities, comparison}. 

\begin{figure}[h]
	\centering
	\begin{subfigure}{0.49\textwidth}
		\includegraphics[width=\textwidth]{/afs/cern.ch/user/d/dvoong/cmtuser/DaVinci_v33r6/Phys/ChargedParticleMultiplicity/python/kinematic_distributions/tracks/results/data_files/manual/real/eta.png}
		\caption{$\eta$}
		\label{fig: charged particle density, eta result}
	\end{subfigure}
	\begin{subfigure}{0.49\textwidth}
		\includegraphics[width=\textwidth]{/afs/cern.ch/user/d/dvoong/cmtuser/DaVinci_v33r6/Phys/ChargedParticleMultiplicity/python/kinematic_distributions/tracks/results/data_files/manual/real/pt.png}
		\caption{$p_\mathrm{T}$}
		\label{fig: charged particle density, pt result}
	\end{subfigure}
	\caption{Corrected charged particle densities, systematic errors are shown by the shaded blue areas}
	\label{fig: corrected charged particle densities}
\end{figure}

\begin{figure}[H]
	\centering
	\begin{subfigure}{0.49\textwidth}
		\includegraphics[width=\textwidth]{/afs/cern.ch/user/d/dvoong/cmtuser/DaVinci_v33r6/Phys/ChargedParticleMultiplicity/python/kinematic_distributions/tracks/results/data_files/manual/real/eta_comparison.png}
		\caption{$\eta$}
		\label{fig: charged particle density, eta result}
	\end{subfigure}
	\begin{subfigure}{0.49\textwidth}
		\includegraphics[width=\textwidth]{/afs/cern.ch/user/d/dvoong/cmtuser/DaVinci_v33r6/Phys/ChargedParticleMultiplicity/python/kinematic_distributions/tracks/results/data_files/manual/real/pt_comparison.png}
		\caption{$p_\mathrm{T}$}
		\label{fig: charged particle density, pt result}
	\end{subfigure}
	\caption{Comparison of corrected charged particle densities between measured data and several MC generators.}
	\label{fig: corrected charged particle densities, comparison}
\end{figure}

%%\chapter{Conclusion} % Write in your own chapter title
%\label{Conclusion}
\chapter{Conclusion}
\lhead{\emph{Conclusion}} % Write in your own chapter title to set the page header

Outlined in this thesis are the key physics underlying particle production phenomena as well as prospective areas of interest in this field of physics. The LHCb detector is highly suited to this are of research due especially to its exceptional tracking as well as its overall performance in all areas. 

A method in which the effective position of Hybrid Photo-Detectors (HPDs) in Cherenkov imaging detectors may be aligned by software was presented. This has shown to have a significant increase in the ability of the detector to distinguish particle species, aiding in many of the key measurements made at the LHCb detector. The particle identification power of the LHCb detector also gives a unique opportunity of studying the charged particle production of individual particle species further constraining current event generator models.

Comparisons between the unfolded charged particle distributions and Monte Carlo event generated distributions show significant differences between measured data and MC simulators (as well as between the different event generators). The measurements in this thesis will hopefully go on to help constrain these generators, increasing their predictive powers well as providing insight into the phenomena of particle production in the soft QCD regime. Understanding these phenomena is particularly important for physics at the LHC since multiplicity sensitive phenomena such additional hard or semi-hard scatters are predicted to be more prevalent at the collision energies present at the LHC \cite{arXiv:1111.0469}.

With the increasing collision energies at luminosity at the LHC, the future at the LHC promises to be an exciting and challenging environment in which great advances will be made.

%\subsection{Uncorrected Distributions}
%
%\subsection{Background Corrected Distributions}
%
%\subsection{Efficiency Corrected Distributions}
