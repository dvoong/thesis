\section{Introduction}
\label{section: multiplicity introduction}

The LHCb detector provides a unique environment in which to study particle multiplicities providing an opportunity to investigate properties of particle production at a unique energy regime and kinematic range with a high level of precision due to the excellent tracking of the detector. The analysis of the production of charged particles is studied as a function of pseudorapidity and transverse momentum. In addition to this the inclusive particle multiplicity is studied for the whole pseudorapidity and momentum range.

%In this analysis the charged particle multiplicity is presented in terms of the charge particle density as a function of pseudorapidity and transverse momentum - giving insight into the regions in which charged particles are produced, and the event multiplicity due to a single proton-proton interaction - giving a broader understanding of the event as a whole.

In this chapter the data selection used is discussed followed by the correction procedures used to remove background contributions. The procedures used to correct detector efficiency effects (unfolding) are then considered followed by an overview of the systematic uncertainties associated to each of the correction procedures. The results are then presented together with comparisons to Monte Carlo event generator predictions. %some of the simulation models currently in use in the  field of particle physics. A discussion of their implications for future measurements in particle physics.
