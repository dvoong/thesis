\section{Prompt Particles}

%Over the time evolution of a proton-proton collision event the multiplicity may change. For example, neutral particles may decay into pairs of oppositely charged particles such as in the case of the decay of the $\mathrm{K}_\mathrm{s}$ meson into a pair of oppositely charged pions. In order to provide a measurement of the charged particle multiplicity that can be used compared to phenomenological models a clear definition of the multiplicity must be made. In this analysis the multiplicity is defined as the number of \textbf{prompt} (defined below) particles produced from the interaction.

Prompt particles are defined as stable particles produced from the initial proton-proton interaction or from the decay of short lived states that are produced in the initial proton-proton interaction. The set of stable charged particles is composed of electrons, muons, charged pions, kaons and protons. A proper lifetime cut of 0.01 nanoseconds is imposed on a decay process such that only stable particles from processes with a combined lifetime less than the cut are considered. Under this definition charged particles from the decay of $K_s$ and $\Lambda$ mesons which have mean lifetimes of $0.08954 \pm 0.00004$ ns and $0.2632 \pm 0.002$ ns respectively, are not classified as prompt particles.

The $\eta$ and $p_\mathrm{T}$ distributions of prompt particles over the full kinematic range (i.e. excluding any detector acceptance cuts) calculated from MC data is shown in figure \ref{fig: generated prompt particle density}.

\begin{figure}[h]
	\centering
	\begin{subfigure}[b]{0.49\textwidth}
		\includegraphics[width=\textwidth]{/afs/cern.ch/user/d/dvoong/cmtuser/DaVinci_v33r6/Phys/ChargedParticleMultiplicity/python/kinematic_distributions/genps/data_files/GenpDistributionsPlottingJob/bk/Down/mc/-1/-1/bk/Down/mc/-1/-1/all_genps/event_selection_genp/meissner/pngs/eta_norm_event.png}
		\caption{$\eta$}
		\label{fig: gen particle density eta}
         \end{subfigure}
	\begin{subfigure}[b]{0.49\textwidth}
		\includegraphics[width=\textwidth]{/afs/cern.ch/user/d/dvoong/cmtuser/DaVinci_v33r6/Phys/ChargedParticleMultiplicity/python/kinematic_distributions/genps/data_files/GenpDistributionsPlottingJob/bk/Down/mc/-1/-1/bk/Down/mc/-1/-1/all_genps/event_selection_genp/meissner/pngs/pt_norm_event.png}
		\caption{$p_\mathrm{T}$}
		\label{fig: gen particle density pt}
         \end{subfigure}
%	\begin{subfigure}[b]{0.32\textwidth}
%		\includegraphics[width=\textwidth]{/afs/cern.ch/user/d/dvoong/cmtuser/DaVinci_v33r6/Phys/ChargedParticleMultiplicity/python/kinematic_distributions/genps/data_files/GenpDistributionsPlottingJob/bk/Down/mc/-1/-1/bk/Down/mc/-1/-1/all_genps/event_selection_genp/meissner/pngs/phi_norm_event.png}
%		\caption{$\phi$}
%		\label{fig: gen particle density phi}
%         \end{subfigure}
         \caption{Prompt particle density from generated MC samples}
	 \label{fig: generated prompt particle density}
\end{figure}